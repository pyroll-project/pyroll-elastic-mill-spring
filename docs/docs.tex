%! Author = Christoph Renzing
%! Date = 21.04.22

% Preamble
\documentclass[11pt]{PyRollDocs}
\usepackage{textcomp}

\addbibresource{refs.bib}
% Document
\begin{document}

    \title{Elastic mill spring PyRoll Plugin}
    \author{Christoph Renzing}
    \date{\today}

    \maketitle

    This plugin calculates the elastic roll stand feedback according to the Gaugemeter equation.


    \section{Model approach}\label{sec:model-approach}

    During rolling, the stand in which the rolls as well as secondary equipment are mounted, is normally treated as rigid.
    Measurements and investigations by \textcite{Weber1973} show, that this is not correct and the whole setup should be viewed as an elastic system of springs.
    Since not all of those values can be measured sufficiently, the elastic response of the whole stand is modeled by a single spring constant ($C$).
    This constant is depended on the roll force ($F_W$) which is applied and therefore leads to a offset between set ($s_0$) and real ($s_1$) roll gap.
    This can be modeled according to the Gaugemeter equation.

    \begin{equation}
        s_1 = s_0 + \frac{F_W}{C_S}
        \label{eq:gaugemeter-equation}
    \end{equation}


    \section{Usage instructions}\label{sec:usage-instructions}

    The plugin can be loaded under the name \texttt{pyroll\_elastic\_mill\_spring}.

    An implementation of the hook \lstinline{gap} is provided on the \lstinline{RollPass}.
    Furthermore, a hook calculating the \lstinline{roll_gap_offset} is provided.
    Several additional hooks on \lstinline{RollPass} are defined, which are used for calculation, as listed in \autoref{tab:hookspecs}.

    \begin{table}
        \centering
        \caption{Hooks specified by this plugin.}
        \label{tab:hookspecs}
        \begin{tabular}{ll}
            \toprule
            Hook name                       & Meaning                              \\
            \midrule
            \texttt{mill\_stand\_stiffness} & Spring coefficient of the stand  $C_S$ \\
            \bottomrule
        \end{tabular}
    \end{table}

    \printbibliography


\end{document}